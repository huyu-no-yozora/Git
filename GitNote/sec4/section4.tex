\documentclass[10pt,a4j,openany,dvipdfmx]{jsarticle}
\usepackage{docmute} %inputコマンドを使うためのパッケージ
\input{preamble}

\begin{document}

%%%%%%%%%%%%%%%%%%%%%%%%%%%%%%%%%%%%%%%%%%%%%section%%%%%%%%%%%%%%%%%%%%%%%%%%%%%%%%%%%%%%%%%%%%%%%%
\section{Deep Dive} % (fold)
\label{sec:Deep Dive}


%%%%%%%%%%%%%%%%%%%%%%%%%%%%%%%%%%%%%%%%%%%%%%%%%%%%
\subsection{役に立つコマンド} % (fold)
\label{sub:役に立つコマンド}


\subsubsection{cherry-pick} % (fold)
\label{ssub:cherry_pick}

他のブランチなどにあるcommitも取り込むことができる。\\
取り込みたいbranchにcheckoutし、その後、取り込みたいcommit idを羅列するだけで良い。
\begin{commandshell}
git checkout [target branch]
git cherry-pick [target commit 1] [target commit 2] ...
\end{commandshell}

詳細は以下を参照。
cherry-pick: \url{https://git-scm.com/docs/git-cherry-pick}

% subsubsection cherry_pick (end)
\subsubsection{tag} % (fold)
\label{ssub:tag}

特定のcommitに対してv1.0というタグをつけたい場合は以下のように実行する。
\begin{commandshell}
git tag v1.0 [target commit id]
git tag -m "version 1.0" v1.0 [target commit id]   # with a comment
\end{commandshell}

現在のタグを確認する場合は、以下を実行すれば良い。
\begin{commandshell}
git tag
\end{commandshell}

詳細は、以下を参照。
tag: \url{https://git-scm.com/book/en/v2/Git-Basics-Tagging}

% subsubsection tag (end)
\subsubsection{show} % (fold)
\label{ssub:show}

特定のcommitの変更内容を表示できる。最新のcommitの変更内容を表示するには以下を実行する。
\begin{commandshell}
git show HEAD
\end{commandshell}

% subsubsection show (end)
\subsubsection{clean} % (fold)
\label{ssub:clean}

\begin{commandshell}
git clean -n
git clean -f 
\end{commandshell}

% subsubsection clean (end)
\subsubsection{archive} % (fold)
\label{ssub:archive}

\$ git archive --format=tar --prefix=[directory name] HEAD | \
gzip > [directory path]/[file name].tar.gz

prefixオプションで、展開後のディレクトリを指定できる。

% subsubsection archive (end)
\subsection{hunk単位でのオペレーション} % (fold)
\label{sub:hunk単位でのオペレーション}

\begin{tcolorbox}[title=hunkとは, fonttitle=\bfseries]
管理対象のファイルに対し、Gitはブロック単位でコードを認識している。
hunkとは、コード内におけるGitが認識するブロックのことである。
\end{tcolorbox}

以下ではhunk単位での扱いについて記載する。

hunk単位でのindexへの登録
\begin{commandshell}
git add -p
\end{commandshell}

hunk単位でのcommit
\begin{commandshell}
git commit --interactive
\end{commandshell}

% subsection hunk単位でのオペレーション (end)

% subsection 役に立つコマンド (end)
%%%%%%%%%%%%%%%%%%%%%%%%%%%%%%%%%%%%%%%%%%%%%%%%%%%%
\subsection{もっと学びたい方へ} % (fold)
\label{sub:もっと学びたい方へ}


\begin{skybox}{Git Branch Model}
Git Branch Modelというものを学習すると良いだろう。
Git-Flow, GitHub-Flowなどが存在する。
\tcblower
\url{https://nvie.com/posts/a-successful-git-branching-model/}
\end{skybox}

\begin{redbox}{ProGit}
ProGitという本がGit本家から出ているので、より詳しく学びたい場合は一読すると良いだろう。
\tcblower
Git HP: \url{https://git-scm.com/book/en/v2}
GitHub: \url{https://github.com/progit/}
\end{redbox}


% subsection もっと学びたい方へ (end)
%%%%%%%%%%%%%%%%%%%%%%%%%%%%%%%%%%%%%%%%%%%%%%%%%%%%


% section Deep Dive (end)
%%%%%%%%%%%%%%%%%%%%%%%%%%%%%%%%%%%%%%%%%%%%%section%%%%%%%%%%%%%%%%%%%%%%%%%%%%%%%%%%%%%%%%%%%%%%%%
\end{document}


