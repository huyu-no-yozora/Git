\documentclass[10pt,a4j,openany,dvipdfmx]{jsarticle}
\usepackage{docmute} %inputコマンドを使うためのパッケージ
\input{preamble}

\begin{document}

%%%%%%%%%%%%%%%%%%%%%%%%%%%%%%%%%%%%%%%%%%%%%section%%%%%%%%%%%%%%%%%%%%%%%%%%%%%%%%%%%%%%%%%%%%%%%%
\section{Deep Dive} % (fold)
\label{sec:Deep Dive}


%%%%%%%%%%%%%%%%%%%%%%%%%%%%%%%%%%%%%%%%%%%%%%%%%%%%
\subsection{役に立つコマンド} % (fold)
\label{sub:役に立つコマンド}


\subsubsection{cherry-pick} % (fold)
\label{ssub:cherry_pick}

cherry-pick: \url{https://git-scm.com/docs/git-cherry-pick}

% subsubsection cherry_pick (end)


% subsection 役に立つコマンド (end)
%%%%%%%%%%%%%%%%%%%%%%%%%%%%%%%%%%%%%%%%%%%%%%%%%%%%
\subsection{もっと学びたい方へ} % (fold)
\label{sub:もっと学びたい方へ}


\begin{skybox}{Git Branch Model}
Git Branch Modelというものを学習すると良いだろう。
Git-Flow, GitHub-Flowなどが存在する。
\tcblower
\url{https://nvie.com/posts/a-successful-git-branching-model/}
\end{skybox}


% subsection もっと学びたい方へ (end)
%%%%%%%%%%%%%%%%%%%%%%%%%%%%%%%%%%%%%%%%%%%%%%%%%%%%


\tcbset{gitexample/.style={listing and comment,comment={#1},
skin=bicolor,boxrule=1mm,fonttitle=\bfseries,coltitle=black,
frame style={draw=black,left color=Gold,right color=Goldenrod!50!Gold},
colback=black,colbacklower=Goldenrod!75!Gold,
colupper=white,collower=black,
listing options={language={bash},aboveskip=0pt,belowskip=0pt,nolol,
basicstyle=\ttfamily\bfseries,extendedchars=true}}}
\begin{tcblisting}{title={Snapshot of the staging area},
gitexample={The option '-a' automatically stages all tracked and modified
files before the commit.\par
This can be combined with the message option '-m'
as seen in the third line.}}
git commit
git commit -a
git commit -am 'changes to my example'
\end{tcblisting}


\begin{myfilebox}[colbacktitle=RoyalBlue]{My title}
\lipsum[2]
\end{myfilebox}


\begin{table}[H]
  \centering
  \begin{tabular}{|c|c|} \hline
      parameter name & meaning\\ \hline \hline
      xlon & longitude\\
      xlat & latitude\\
      ght  & geopotential height\\
      prs  & pressure\\
      www  & \\
      qvp  & qv\\
      rhu  & relative humidity\\ 
      eth  & equivalent potential temperature\\ \hline
  \end{tabular}
\end{table}


% section Deep Dive (end)
%%%%%%%%%%%%%%%%%%%%%%%%%%%%%%%%%%%%%%%%%%%%%section%%%%%%%%%%%%%%%%%%%%%%%%%%%%%%%%%%%%%%%%%%%%%%%%
\end{document}


