% MIT License
% 
% sec1/section1.tex
% 
% Copyright (c) 2020 冬ノ夜空
% 

\documentclass[10pt,a4j,openany,dvipdfmx]{jsarticle}
\usepackage{docmute} %for "input" command
\input{preamble}

\begin{document}

% \section*{secname}
% \addcontentsline{toc}{section}{secname}

%%%%%%%%%%%%%%%%%%%%%%%%%%%%%%%%%%%%%%%%%%%%%section%%%%%%%%%%%%%%%%%%%%%%%%%%%%%%%%%%%%%%%%%%%%%%%%
\section{Gitとは}

%@ Git =====================================================%
\subsection{Gitとは} % (fold)
\label{sub:gitとは}

\begin{tcolorbox}[enhanced,colback=gray!5!white,
title=Gitについて,coltitle=black,fonttitle=\bfseries,
frame style image=blueshade.png,
overlay={\begin{tcbcliptitle}\node at (title)
{\includegraphics[width=\linewidth]{./figure/beautiful_flower.png}};\end{tcbcliptitle}},
watermark opacity=0.1, watermark text app=Git, watermark color=Gray, watermark zoom=0.5]
Git is a free and open source distributed version control system designed to handle everything from small to very large projects with speed and efficiency.\ (Git\ HP)
\tcblower
HP:\ \url{https://git-scm.com/}
\end{tcolorbox}

% subsection gitとは (end)
%@ Git =====================================================%

%@ 環境設定 =================================================%
\subsection{環境設定} % (fold)
\label{sub:環境設定}


システムへのインストール後、以下コマンドを実行する。\\

global設定
\begin{commandshell}
git config --global user.name "[your name]"
git config --global user.email "[your email address]"
\end{commandshell}

なお、特定のrepositoryにのみ別の設定を反映したい場合には、以下のコマンドを実行する。
\begin{commandshell}
cd "[path to a local repository]"
git config --local user.name "[your name]"
git config --local user.email "[your email address]"
\end{commandshell}


% subsection 環境設定 (end)
%@ 環境設定 =================================================%

%@ GitHub ==================================================%
\subsection{関連サービス} % (fold)
\label{sub:関連サービス}


\begin{picturebox}[title=GitHubについて]{./figure/universe.jpg}
GitHub is a code hosting platform for version control and collaboration. It lets you and others work together on projects from anywhere.\\
HP:\ \url{https://github.com/}
\tcblower
GitHub Offical: Feature\\
\url{https://github.com/features}
\end{picturebox}

一言で言えば
\begin{tcolorbox}[enhanced, fuzzy halo=0.5mm with Gray, frame style image=blueshade.png]
ソースコードをリモートに保存し、必要に応じて公開することができるサービス
\end{tcolorbox}
である。CI/CDを組んだりできるようになったことから、インフラ基盤として利用されることも多い。


GitHubのチュートリアル\\
\url{https://guides.github.com/activities/hello-world/}


% subsection 関連サービス (end)
%@ GitHub ==================================================%

\end{document}


