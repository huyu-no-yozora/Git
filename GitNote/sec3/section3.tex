\documentclass[10pt,a4j,openany,dvipdfmx]{jsarticle}
\usepackage{docmute} %inputコマンドを使うためのパッケージ
\input{preamble}

\begin{document}

%%%%%%%%%%%%%%%%%%%%%%%%%%%%%%%%%%%%%%%%%%%%%section%%%%%%%%%%%%%%%%%%%%%%%%%%%%%%%%%%%%%%%%%%%%%%%%
\section{基礎コマンド} % (fold)
\label{sec:基礎コマンド}


%%%%%%%%%%%%%%%%%%%%%%%%%%%%%%%%%%%%%%%%%%%%%%%%%%%%
\subsection{branchの作成とcheckout} % (fold)
\label{sub:branchの作成とcheckout}

\begin{redbox}{checkout}
checkoutとは、
\tcblower
どこにcheckoutしているのか知るために、以下の環境変数を変更しておくと良いかもしれない。
\end{redbox}

まず、以下コマンドでbranchを作成する。
\begin{commandshell}
git branch [branch name]
\end{commandshell}

作成したbranchにcheckoutしてみる。
\begin{commandshell}
git checkout [branch name]
\end{commandshell}


また、以下のコマンドで、上記のローカルbranchの作成とcheckoutをまとめて行える。
\begin{commandshell}
git checkout -b [branch name]
\end{commandshell}

% subsection branchの作成とcheckout (end)
%%%%%%%%%%%%%%%%%%%%%%%%%%%%%%%%%%%%%%%%%%%%%%%%%%%%
\subsection{リモートリポジトリへの変更の送信の流れ} % (fold)
\label{sub:リモートリポジトリへの変更の送信の流れ}


リモートリポジトリへの変更のアップロード手順に関しては、主に以下のStepから成る。
\begin{tcolorbox}[enhanced,title=Git\ Main\ Process,fonttitle=\bfseries,
arc=3mm, arc is curved, top=5mm, bottom=2mm, left=10mm, 
colframe=RoyalBlue!80!Snow,
colback=White,colbacktitle=White,coltitle=black,
frame style={upper left=Red,upper right=Yellow,lower left=RoyalBlue,lower right=DarkMagenta},
attach boxed title to top center=
{yshift=-0.25mm-\tcboxedtitleheight/2,yshifttext=2mm-\tcboxedtitleheight/2},
boxed title style={boxrule=0.5mm,
frame code={ \path[tcb fill frame] ([xshift=-4mm]frame.west)
-- (frame.north west) -- (frame.north east) -- ([xshift=4mm]frame.east)
-- (frame.south east) -- (frame.south west) -- cycle; },
interior code={ \path[tcb fill interior] ([xshift=-2mm]interior.west)
-- (interior.north west) -- (interior.north east)
-- ([xshift=2mm]interior.east) -- (interior.south east) -- (interior.south west)
-- cycle;} }]
Step1.\ git addによるインデックスへのファイルの追加等\\
Step2.\ git commitによるインデックスへの変更の反映\\
Step3.\ git pushによる変更内容のリモートリポジトリへのアップロード
\end{tcolorbox}

% \begin{tcolorbox}[enhanced, arc=3mm, arc is curved, fuzzy halo=0.5mm with Gray!80!White, frame style image=blueshade.png]
% \textbtt{Git\ Main\ Process}
% \tcblower
% Step1.\ git addによるインデックスへのファイルの追加等
% Step2.\ git commitによるインデックスへの変更の反映
% Step3.\ git pushによる変更内容のリモートリポジトリへのアップロード
% \end{tcolorbox}

以下のようなファイルをGitで管理したいとする。
\begin{numList}{c}
#include <stdio.h>
int main() {
   // printf() displays the string inside quotation
   printf("Hello, World!");
   return 0;
}
\end{numList}

\subsubsection{Step1. add - indexに登録する} % (fold)
\label{ssub:step1_add_indexに登録する}

以下コマンドを実行し、Gitの管理下にあるファイルをindexに追加する。
\begin{commandshell}
git add .
\end{commandshell}

以下のように編集したとする。
\begin{numList}{c}
#include <stdio.h>
int main() {
   // printf() displays the string inside quotation
   printf("Good Bye!");
   return 0;
}
\end{numList}

再度\commandbox{git add .}を実行し、indexに追加する。


% subsubsection step1_add_indexに登録する (end)
\subsubsection{Step2. commit - indexに変更を反映する} % (fold)
\label{ssub:step2_commit_indexに変更を反映する}

以下コマンドを実行し、変更内容をindexに反映(保存)する。
\begin{commandshell}
git commit -m "[comment]"
\end{commandshell}

リモートに反映させたい修正の範囲になるまで、Step1, Step2を繰り返す。

% subsubsection step2_commit_indexに変更を反映する (end)
\subsubsection{Step3. push - リモートへの変更内容の送信} % (fold)
\label{ssub:step3_push_リモートへの変更内容の送信}

以下コマンドを実行し、リモートに変更内容を送信する。
\begin{commandshell}
git push origin [branch name]
\end{commandshell}

% subsubsection step3_push_リモートへの変更内容の送信 (end)


% subsection リモートリポジトリへの変更の送信の流れ (end)
%%%%%%%%%%%%%%%%%%%%%%%%%%%%%%%%%%%%%%%%%%%%%%%%%%%%
\subsection{その他の使用頻度の高いコマンド} % (fold)
\label{sub:その他の使用頻度の高いコマンド}

\subsubsection{リモートの変更内容の反映} % (fold)
\label{ssub:リモートの変更内容の反映}

\begin{commandshell}
git pull
\end{commandshell}

% subsubsection リモートの変更内容の反映 (end)
\subsubsection{状態の確認} % (fold)
\label{ssub:状態の確認}

ローカルリポジトリ、ワーキングツリーの状態の確認
\begin{commandshell}
git status
\end{commandshell}

コミット履歴等の確認
\begin{commandshell}
git log
\end{commandshell}

% subsubsection 状態の確認 (end)
\subsubsection{変更の取り消し(ローカルブランチに対してのみ推奨)} % (fold)
\label{ssub:変更の取り消し_ローカルブランチに対してのみ推奨_}

\commandbox{git reset}コマンドでは、特定のコミットによる変更を戻すことができる。
なお、デフォルトのオプションは\commandbox{--mixed}(HEADとインデックスのリセット)である。

ここでは、ローカルリポジトリのワーキングツリーやインデックス、ブランチを最新のコミットの内容にまで戻す場合について記載する。
\begin{commandshell}
git reset --hard HEAD
\end{commandshell}

または、ワーキングツリーの内容を最新のコミットの状態に戻すには、以下のコマンドでも良い。
\begin{commandshell}
git stash
git checkout HEAD
\end{commandshell}

% subsubsection 変更の取り消し_ローカルブランチに対してのみ推奨_ (end)
\subsubsection{一時保存} % (fold)
\label{ssub:一時保存}

\begin{commandshell}
git stash (save [saving name]) 
\end{commandshell}

保存されているキューの一覧の表示
\begin{commandshell}
git stash list
\end{commandshell}

キューに保存された内容の表示
\begin{commandshell}
git stash show -p stash@{[queue number]}
\end{commandshell}

保存したキューの内容を適用
\begin{commandshell}
git stash apply [stash@{stash[queue number]} or saved name]
\end{commandshell}

保存されたキューの内容を削除
\begin{commandshell}
git stash drop [stash@{stash[queue number]} or saved name]
\end{commandshell}

% subsubsection 一時保存 (end)
\subsubsection{Rebase} % (fold)
\label{ssub:rebase}



feature/exampleブランチをdevelopブランチに追従させる場合
\begin{filecontents}{rebase_before.txt}
* cfb44f5 (HEAD -> develop) add description about hunk
* 69119b1 add description of some useful command
| * be8634b (feature/example) add description about how to delete a branch
| * e5ad949 change a subsubsection name
|/  
* 561e128 add a Contents table
* 18d3438 (master) initial code was putted
\end{filecontents}

\begin{tikzpicture}
\tikzstyle{commit}=[draw,circle,fill=white,inner sep=0pt,minimum size=5pt]
\tikzstyle{clabel}=[right,outer sep=1em]
\tikzstyle{every path}=[draw]
\matrix [column sep={1em,between origins},row sep=\lineskip]
{
\commit{cfb44f5}{(develop) add description about hunk} & \\
\commit{69119b1}{add description of some useful command} & \\
  & \commit{be8634b}{(feature/example) add description about how to delete a branch} \\
  & \commit{e5ad949}{change a subsubsection name} \\
\commit{561e128}{add a Contents table} & \\
\commit{18d3438}{(master) initial code was putted} & \\
};
\connect{18d3438}{561e128};
\connect{561e128}{69119b1};
\connect{69119b1}{cfb44f5};
\connect{561e128}{e5ad949};
\connect{e5ad949}{be8634b};
\end{tikzpicture}


以下を実行する。
\begin{commandshell}
git checkout feature/example
git rebase develop 
\end{commandshell}


\begin{filecontents}{rebase_after.txt}
  * ef963ca (HEAD -> feature/example) add description about how to delete a branch
  * bfb0835 change a subsubsection name
 /  
* cfb44f5 (develop) add description about hunk
* 69119b1 add description of some useful command
* 561e128 add a Contents table
* 18d3438 (master) initial code was putted
\end{filecontents}

\begin{tikzpicture}
\tikzstyle{commit}=[draw,circle,fill=white,inner sep=0pt,minimum size=5pt]
\tikzstyle{clabel}=[right,outer sep=1em]
\tikzstyle{every path}=[draw]
\matrix [column sep={1em,between origins},row sep=\lineskip]
{
  & \commit{ef963ca}{(HEAD -> feature/example) add description about how to delete a branch} \\
  & \commit{bfb0835}{change a subsubsection name} \\
\commit{cfb44f5}{(develop) add description about hunk} & \\
\commit{69119b1}{add description of some useful command} & \\
\commit{561e128}{add a Contents table} & \\
\commit{18d3438}{(master) initial code was putted} & \\
};
\connect{18d3438}{561e128};
\connect{561e128}{69119b1};
\connect{69119b1}{cfb44f5};
\connect{cfb44f5}{bfb0835};
\connect{bfb0835}{ef963ca};
\end{tikzpicture}


% subsubsection rebase (end)
\subsubsection{Merge} % (fold)
\label{ssub:merge}

\begin{commandshell}
git checkout [source branch]

# 3 way - merge
git merge --no-ff [target branch]

# fast forward merge (default)
git merge [target branch]
\end{commandshell}


% gitgraph.txt contains raw output of: $ git log --graph --oneline
\begin{filecontents}{gitgraph.txt}
* d764b48 added plaintext version in markdown
* 54ba4b2 release 2014-01-25
*   c589395 Merge branch 'master'
|\
| * 9f9c652 Remove holdover from kjh gh-pages branch
* | b3bd158 exclude font files
|/
* 63268c1 micro-typography
\end{filecontents}

\begin{tikzpicture}
\tikzstyle{commit}=[draw,circle,fill=white,inner sep=0pt,minimum size=5pt]
\tikzstyle{clabel}=[right,outer sep=1em]
\tikzstyle{every path}=[draw]
\matrix [column sep={1em,between origins},row sep=\lineskip]
{
\commit{d764b48}{added plaintext version in markdown} & \\
\commit{54ba4b2}{release 2014-01-25} & \\
\commit{c589395}{Merge branch `master'} & \\
 & \commit{9f9c652}{Remove holdover from kjh gh-pages branch} \\
\commit{b3bd158}{exclude font files} & \ghost{branch1} \\
\commit{63268c1}{micro-typography} & \\
};
\connect{63268c1}{b3bd158};
\connect{63268c1}{branch1};
\connect{branch1}{9f9c652};
\connect{b3bd158}{c589395};
\connect{9f9c652}{c589395};
\connect{c589395}{54ba4b2};
\connect{54ba4b2}{d764b48};
\end{tikzpicture}


\begin{oceanbox}{3 way - merge と fast forward merge の使い分け}
With the 'enhanced' skin, it is quite easy to produce fancy looking effects.
\tcblower
Note that ...
\end{oceanbox}


\begin{oceanbox}{rebase と merge の使い分け}
With the 'enhanced' skin, it is quite easy to produce fancy looking effects.
\tcblower
Note that ...
\end{oceanbox}

% subsubsection merge (end)
\subsubsection{コミットメッセージの変更} % (fold)
\label{ssub:コミットメッセージの変更}

直前のcommit messageを変更するには以下を実行すれば良い。
\begin{commandshell}
git commit --amend
\end{commandshell}

2つ以上前のcommit messageを変更する場合には以下のようになる。

\begin{tcolorbox}[skin=enhanced,
left=3mm,right=3mm,top=1mm,bottom=1mm, boxrule=1.0mm, 
title=2つ以上前のcommit messageの変更, coltitle=black, fonttitle=\bfseries, 
colback=SpringGreen!5!white,colframe=SpringGreen!70]
n個前のcommit messageを変更したいとする。\\
\\
Step1. 変更対象のbranchにcheckoutし、以下を実行する
\begin{commandshell}
git rebase --interactive HEAD~n
\end{commandshell}

Step2. 変更したい対象のcommitのcommit idの隣の「pick」という文字を「edit」に書き換え
最初のeditの部分に戻っているので、以下を実行する
\begin{commandshell}
git commit --amend
\end{commandshell}

Step3. commit messageを変更後、以下を実行する
\begin{commandshell}
git rebase --continue
\end{commandshell}

Step4. 以降、editに書き換えた部分に対し、Step2, Step3を繰り返す\\
\\
Step5. git logを実行し、変更が反映されていることを確認する
\end{tcolorbox}


% subsubsection コミットメッセージの変更 (end)
\subsubsection{ブランチの削除} % (fold)
\label{ssub:ブランチの削除}

branchの削除の仕方については以下の2種類がある。\\
元になっているブランチよりも新しいcommitがある場合には、dオプションではerrorを返す。
\begin{commandshell}
git branch -d [target branch]
git branch -D [target branch]   # by force
\end{commandshell}

% subsubsection ブランチの削除 (end)


% subsection その他の使用頻度の高いコマンド (end)
%%%%%%%%%%%%%%%%%%%%%%%%%%%%%%%%%%%%%%%%%%%%%%%%%%%%


% section 基礎コマンド (end)
%%%%%%%%%%%%%%%%%%%%%%%%%%%%%%%%%%%%%%%%%%%%%section%%%%%%%%%%%%%%%%%%%%%%%%%%%%%%%%%%%%%%%%%%%%%%%%
\end{document}


