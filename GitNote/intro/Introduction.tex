% MIT License
% 
% intro/Introduction.tex
% 
% Copyright (c) 2020 冬ノ夜空
% 

\documentclass[10pt,a4j,openany,dvipdfmx]{jsarticle}
\usepackage{docmute} %for "input" command
\input{preamble}

\begin{document}
\section*{\it {~\ Introduction\ ~}}
\addcontentsline{toc}{section}{\it {~\ Introduction\ ~}}
%%%%%%%%%%%%%%%%%%%%%%%%%%%%%%%%%%%%%%%%%%%%%%%%%%%%%%%%%%%%%%


% document全体に適応
% \CenterWallPaper{scaling}{filename}
% 単一ページのみ適応
% \ThisCenterWallPaper{scaling}{filename}


\ThisCenterWallPaper{1}{./figure/intro.png}
% \ThisLRCornerWallPaper{1}{./figure/colorful.png}

Gitとは、一言で言えば\textcolor{RoyalBlue}{ ファイル群のバージョン管理を行うための仕組み }である。
Linuxの生みの親であるLinus Torvalsによって作成された。現在でも、日々アップデートが続いている。
現在では、主に、プログラムの開発者が共同でコーディングを行い開発作業を安定して進める際に使用される。
このドキュメントでは、Gitの使い方を簡単に学ぶためのガイド的な立ち位置のもとに、Gitを使用するための基礎的な事項についてのみ記載する。
Gitの便利なコマンドやより深い内容などといった、より詳細なインフォメーションは、公式ドキュメントや本などで学習すると良いだろう。


%%%%%%%%%%%%%%%%%%%%%%%%%%%%%%%%%%%%%%%%%%%%%%%%%%%%%%%%%%%%%%
\end{document}


